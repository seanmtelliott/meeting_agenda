%%% Document class (e.g. article, letter, beamer, etc...)
\documentclass{article}
%%%

%%% Packages
\usepackage[round]{natbib} %Allows custom options for bibliography management
\usepackage{cite} %Adds bibliograhy functionality like \citet
\usepackage{geometry} %Allows me to set the margins
\usepackage{amsmath} %Adds additional math-related commands
\usepackage{amsfonts} %Adds fonts-within-math-mode functionality
\usepackage{amsthm} %Enables custom theorem environments
\usepackage{bm} %Adds bold math fonts
\usepackage{setspace} %Enables \doublespacing command
\usepackage{parskip} %Removes indentation
\usepackage{tikz} %Used for formatting plots and figures
\usepackage{float} %Used to force the position of tables and figures
\usepackage{graphicx} %For including images
\usepackage[T1]{fontenc} %Allows for encoding of slavic letters
\usepackage{lmodern} %Need to load back the original font for when T1 encoding is active
%%%

%%% Set the margins of the document
\geometry{
 left=1in,
 right=1in,
 bottom=1in,
 top=1in,
 }
%%%
 
%%% User-defined commands

\DeclareMathOperator*{\argmax}{argmax} % The argmax operator

\newtheorem{assumption}{Assumption} % Treating assumptions like theorems.
\newtheorem{proposition}{Proposition} % Treating propositions like theorems.

%%%

%%% Title

%%%

\begin{document}
%\doublespacing
%\interfootnotelinepenalty=10000

\section*{Notes for Jan 8 Meeting}
\subsection*{What I have done:}

\begin{itemize}
\item \textbf{My plan for the intro chapter}
\begin{itemize}
\item Discuss the data in detail: what can we measure, what can't we measure, which countries have reliable/useful data, etc.
\item Cover the second demographic transition and discuss changes in: marriage rate vs. cohabitation rate, divorce and separation, fertility, matching patterns (e.g., is there more or less positive assortative matching?) -- display stylized facts on these topics.
\item Literature review on matching and separation.
\item Relate the matching and demographic literature with some illustration of how one of the extant matching models can be used to say something about demographic transition.

\end{itemize}

\item \textbf{Some reading on TU matching/separation}
\begin{itemize}
\item Aloysius asked me to read and provide a referee report for a paper Bernard Salani\'{e} sent him. It is the chapter on TU matching for the upcoming Handbook of Matching - more or less a summary of \citet{galichon2022cupid} and \citet{galichon2022estimating} using the Logit assumption from \citet{choo2006marries} for illustration purposes.
\item \citet{adda2020there} endogenize separation in a TU context using a match-specific shock - an interesting starting point but needs to be fleshed out.
\item An interesting identification technique in \citet{gualdani2023partial} which may be useful.
\end{itemize}


\item \textbf{Started writing down a model for separation}
\begin{itemize}
\item The TU framework was fresh in my mind from reading Bernard's paper so I started there.
\item I assume everyone is matched and we can identify/observe the average surplus of a match of some given type which is split between the individuals matched.
\item A match dissolves if match-specific shocks cause the surplus falls below a threshold and there is an associated cost of separation.
\item We would like to identify this cost and threshold. 
\item I will bring a print out with some details.
\end{itemize}



\end{itemize}

\subsection*{What is still left to do:}
\begin{itemize}
\item \textbf{Work on the intro chapter}
\begin{itemize}
\item Gather some more data.
\item Put together the demographic transition and matching sections. Most of this is already written -- just need to organize/edit it.
\item Need to think of an interesting application to illustrate the use of matching models in demography.
\end{itemize}
\item \textbf{The model for separation}
\begin{itemize}
\item Keep working on this and add some more details.
\item Incorporate any feedback you have.
\end{itemize}
\end{itemize}

\newpage
\bibliographystyle{plainnat}
\bibliography{/home/selliott/Research/bib/matching}

\end{document}