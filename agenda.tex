%%% Document class (e.g. article, letter, beamer, etc...)
\documentclass{article}
%%%

%%% Packages
\usepackage[round]{natbib} %Allows custom options for bibliography management
\usepackage{cite} %Adds bibliograhy functionality like \citet
\usepackage{geometry} %Allows me to set the margins
\usepackage{amsmath} %Adds additional math-related commands
\usepackage{amsfonts} %Adds fonts-within-math-mode functionality
\usepackage{amsthm} %Enables custom theorem environments
\usepackage{bm} %Adds bold math fonts
\usepackage{setspace} %Enables \doublespacing command
\usepackage{parskip} %Removes indentation
\usepackage{tikz} %Used for formatting plots and figures
\usepackage{float} %Used to force the position of tables and figures
\usepackage{graphicx} %For including images
\usepackage[T1]{fontenc} %Allows for encoding of slavic letters
\usepackage{lmodern} %Need to load back the original font for when T1 encoding is active
%%%

%%% Set the margins of the document
\geometry{
 left=1in,
 right=1in,
 bottom=1in,
 top=1in,
 }
%%%
 
%%% User-defined commands

\DeclareMathOperator*{\argmax}{argmax} % The argmax operator

\newtheorem{assumption}{Assumption} % Treating assumptions like theorems.
\newtheorem{proposition}{Proposition} % Treating propositions like theorems.

%%%

%%% Title

%%%

\begin{document}
%\doublespacing
%\interfootnotelinepenalty=10000

\section*{Notes for March 13 Meeting}
\subsection*{Updates:}

\begin{itemize}
\item \textbf{The intro chapter}
\begin{itemize}
\item I have been working on this and I'm thinking it could be a full chapter, not just an intro chapter.
\item I'd like to revisit measuring cohabitation prior to the CPS including it as an option on the questionnaire in 2007 -- I think there could likely be either some ML or econometric techniques which could be applied.
\item At present, the technique is to use the ``person's of opposite sex living together'' as a proxy for cohab -- is it possible to do better than this? 
\item I could try some predictive model and use the more recent data for cross-validation purposes.
\item Use Ismael and Aloysius' model on this richer data set to explore the differing trends in marriage/cohabitation over the last several decades through the context of gains to each type of relationship within different subgroups of the population.
\item Explore PAM in this context also -- does it exist for cohabitation to the same extent it does for marriage?

\end{itemize}

\item \textbf{The model for divorce}
\begin{itemize}
\item I found a similar example of what I am trying to do in one of Chiappori's old working papers -- a much simpler and static case, but it has some useful information in it I can use.
\item The utility function I use implies TU and we disregard bargaining weights in marriage. Weights in divorce could be fixed by legislation or at the time they match (i.e. prenuptial agreements).
\item The key thing to distinguish between is whether the public good stays public outside of marriage (e.g. children) or if it becomes private (e.g. housing). This influences the shape of the pareto frontier outside of marriage.
\item What I'm struggling with: Suppose I have a social planner's solution, is this guaranteed to also be an equilibrium in a decentralized environment? 
\end{itemize}


\item \textbf{Some other things}
\begin{itemize}
\item Bob reached out to me a little while ago about potentially teaching ECO372 this summer (the May-June session). The role is now officially posted so I was going to apply.
\item I applied for the 2024 Price Theory Summer Camp at the University of Chicago - Becker Friedman Institute and was accepted (occurs in the last week of June).
\end{itemize}



\end{itemize}

\subsection*{Next steps:}
\begin{itemize}
\item \textbf{The ``intro'' chapter}
\begin{itemize}
\item Lots of collecting/cleaning/organizing the data.
\item Start looking into predictive latent variable models. Do household characteristics (aside from just opposite-sex individuals living together) indicate cohabitation?
\end{itemize}
\item \textbf{The model for divorce}
\begin{itemize}
\item I need to think a bit more about the application. 
\item What I am thinking: if I am able to identify entry and exit costs, can I explain some of the recent declines in marriage and increases in cohabitation rates through changes to these costs?
\item Start writing the paper.

\end{itemize}
\end{itemize}

\newpage
\bibliographystyle{plainnat}
%\bibliography{/home/selliott/Research/bib/matching}

\end{document}