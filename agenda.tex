%%% Document class (e.g. article, letter, beamer, etc...)
\documentclass{article}
%%%

%%% Packages
\usepackage[round]{natbib} %Allows custom options for bibliography management
\usepackage{cite} %Adds bibliograhy functionality like \citet
\usepackage{geometry} %Allows me to set the margins
\usepackage{amsmath} %Adds additional math-related commands
\usepackage{amsfonts} %Adds fonts-within-math-mode functionality
\usepackage{amsthm} %Enables custom theorem environments
\usepackage{bm} %Adds bold math fonts
\usepackage{setspace} %Enables \doublespacing command
\usepackage{parskip} %Removes indentation
\usepackage{tikz} %Used for formatting plots and figures
\usepackage{float} %Used to force the position of tables and figures
\usepackage{graphicx} %For including images
\usepackage[T1]{fontenc} %Allows for encoding of slavic letters
\usepackage{lmodern} %Need to load back the original font for when T1 encoding is active
%%%

%%% Set the margins of the document
\geometry{
 left=1in,
 right=1in,
 bottom=1in,
 top=1in,
 }
%%%
 
%%% User-defined commands

\DeclareMathOperator*{\argmax}{argmax} % The argmax operator

\newtheorem{assumption}{Assumption} % Treating assumptions like theorems.
\newtheorem{proposition}{Proposition} % Treating propositions like theorems.

%%%

%%% Title

%%%

\begin{document}
%\doublespacing
\interfootnotelinepenalty=10000

\section*{Notes for November 27 Meeting}

\begin{itemize}
\item \textbf{Find the canonical text(s) for demographic transitions in economics}
\begin{itemize}



\item \textit{Population Economics - Razin and Sadka (1995)}: An advanced undergrad or graduate course on population economics.
\item \textit{Population Dynamics - Chu (1998)}: A curated volume of papers in population economics focusing on macro impact of micro decisions - aimed at researchers.

\item The Chu book should be quite useful, but Razin and Sadka is quite short so I will read it as well.

\item Also found a few syllabi for graduate courses in population economics. No textbooks but some papers I added to my reading list.

\end{itemize}

\item \textbf{Look for demographic research in economics:}
\begin{itemize}
\item A handful of papers in the top journals: AER, Journal of Economic Growth, Journal of Politcal Economy. Also a collection of field journals: Journal of Population Economics, Demography, Journal of Demographic Economics. Journal of Economic Perspectives also has a few interesting survey papers.
\item Sociologists tend to produce the most demographic research - these papers provide useful context. 
\item Demographic research in economics largely appears to be trying to structure the descriptive analyses found in the sociology research.
\item Demographers tend to see there being two distinct transitions: from pre-industrial to developed, and then developed to what we are observing today in marriage and fertility trends.
\item This ``second'' demographic transition is the motivation for what I am interested in doing.
\item In 2015 the political economist Nicholas Eberstadt said in a NYT article: ``Long stable marriages are out, and divorce or separation are in, along with serial cohabitation and increasingly contingent liaisons.''
\item We observe significantly more cohabitation relative to the past, and thus, while divorce rates are dropping, separation rates among all types of coresidential pairings are likely making up this difference, but it is hard to measure. 
\end{itemize}


\item \textbf{Stylized Facts about Divorce}
\begin{itemize}
\item The Statistics Canada website crashed last night and did not come back up before I had to go to bed. However, they have a very nice report from 2022 which I can use to put together a list of facts.
\item I also need to look at other countries.
\end{itemize}

\item \textbf{Consider the simplest model of divorce where we only observe hazard rates, and everyone is of the same type and married.}

In this case, I am thinking of divorce in a similar way to how one thinks of probability of default in a credit risk context.


\item \textbf{What happens when a divorce occurs in these data - do they drop out?}
\begin{itemize}
\item Still not sure, I need to see a sample of one of these data sets, which I don't have yet.
\item Based on my reading I'm now also curious about how we could identify separation in a cohabitation setting.
\end{itemize}

\item \textbf{See what other data sets are available (not just Canada)}
\begin{itemize}
\item Still to do.
\end{itemize}

\end{itemize}

\newpage


\textbf{Some of the papers I read:}
\begin{itemize}
\item \textit{Historical Context/First Demographic Transition}:
\begin{itemize}
\item Coale, A. J. (1989). Demographic transition. In \textit{Social Economics} (pp. 16-23). London: Palgrave Macmillan UK.
\item Kirk, D. (1996). Demographic transition theory. \textit{Population Studies}, 50(3), 361-387.
\item Galor, O., \& Weil, D. N. (1999). From Malthusian stagnation to modern growth. \textit{American Economic Review}, 89(2), 150-154.
\item Lee, R. (2002). The demographic transition: three centuries of fundamental change. \textit{Journal of Economic Perspectives}, 17(4), 167-190.
\item Greenwood, J., \& Seshadri, A. (2002). The US demographic transition. \textit{American Economic Review}, 92(2), 153-159.
\item Lam, D. (2011). How the world survived the population bomb: Lessons from 50 years of extraordinary demographic history. \textit{Demography}, 48(4), 1231-1262.
\item Herzer, D., Strulik, H., \& Vollmer, S. (2012). The long-run determinants of fertility: one century of demographic change 1900–1999. \textit{Journal of Economic Growth}, 17, 357-385.
\item Diebolt, C., \& Perrin, F. (2013). From stagnation to sustained growth: the role of female empowerment. \textit{American Economic Review}, 103(3), 545-549.
\item Lee, R. (2015). Becker and the demographic transition. \textit{Journal of Demographic Economics}, 81(1), 67-74.
\item Madsen, J. B., Robertson, P. E., \& Ye, L. (2019). Malthus was right: Explaining a millennium of stagnation. \textit{European Economic Review}, 118, 51-68.
\end{itemize}

\item \textit{Second demographic transition}

\begin{itemize}
\item Lesthaeghe, R. (1995). The second demographic transition in Western countries: An interpretation. \textit{Gender and Family Change in Industrialized Countries}, 17-62.
\item Lesthaeghe, R. (2010). The unfolding story of the second demographic transition. \textit{Population and Development Review}, 36(2), 211-251.
\item Bailey, M. J., Guldi, M. E., \& Hershbein, B. J. (2013). Is there a case for a" Second Demographic Transition"? Three distinctive features of the post-1960 US fertility decline (No. w19599).  \textit{National Bureau of Economic Research}.
\item Zaidi, B., \& Morgan, S. P. (2017). The second demographic transition theory: A review and appraisal. \textit{Annual Review of Sociology}, 43, 473-492.
\item Manning, W. D., Smock, P. J., \& Fettro, M. N. (2019). Cohabitation and marital expectations among single millennials in the US. \textit{Population Research and Policy Review}, 38(3), 327-346.
\item Sassler, S., \& Lichter, D. T. (2020). Cohabitation and marriage: Complexity and diversity in union‐formation patterns. \textit{Journal of Marriage and Family}, 82(1), 35-61.
\item Kearney, M. S., Levine, P. B., \& Pardue, L. (2022). The puzzle of falling US birth rates since the Great Recession. \textit{Journal of Economic Perspectives}, 36(1), 151-176.
\end{itemize}
\end{itemize}

\end{document}