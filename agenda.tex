%%% Document class (e.g. article, letter, beamer, etc...)
\documentclass{article}
%%%

%%% Packages
\usepackage[round]{natbib} %Allows custom options for bibliography management
\usepackage{cite} %Adds bibliograhy functionality like \citet
\usepackage{geometry} %Allows me to set the margins
\usepackage{amsmath} %Adds additional math-related commands
\usepackage{amsfonts} %Adds fonts-within-math-mode functionality
\usepackage{amsthm} %Enables custom theorem environments
\usepackage{bm} %Adds bold math fonts
\usepackage{setspace} %Enables \doublespacing command
\usepackage{parskip} %Removes indentation
\usepackage{tikz} %Used for formatting plots and figures
\usepackage{float} %Used to force the position of tables and figures
\usepackage{graphicx} %For including images
\usepackage[T1]{fontenc} %Allows for encoding of slavic letters
\usepackage{lmodern} %Need to load back the original font for when T1 encoding is active
%%%

%%% Set the margins of the document
\geometry{
 left=1in,
 right=1in,
 bottom=1in,
 top=1in,
 }
%%%
 
%%% User-defined commands

\DeclareMathOperator*{\argmax}{argmax} % The argmax operator

\newtheorem{assumption}{Assumption} % Treating assumptions like theorems.
\newtheorem{proposition}{Proposition} % Treating propositions like theorems.

%%%

%%% Title

%%%

\begin{document}
%\doublespacing
%\interfootnotelinepenalty=10000

\section*{Notes for August 7 Meeting}
\subsection*{General updates:}

\begin{itemize}
\item \textbf{Committee}
\begin{itemize}
\item Met with Jiaying 
\item I will set a meeting up for the four of us (myself, you, Aloysius, and Jiaying) in September when everyone is back in town.
\end{itemize}
\end{itemize}
\subsection*{Project updates:}
\begin{itemize}
\item \textbf{Predicting historical cohabitation}
\begin{itemize}
\item I have been learning about neural networks.
\item There are computationally efficient algorithms (stochastic gradient descent) which should allow me to use an exceptionally large training set.
\item Once I fit the model, I am going to go back and construct estimates of who was cohabitating using past census data.
\item The prevailing opinion is that cohabitation either didn't increase, or actually declined, from 1880-1970. If my model produces something different, it could be a interesting contribution to the demographic literature.
\end{itemize}


\item \textbf{Decline of marriage}
\begin{itemize}
\item I am trying to assess possible explanations for the decline in marriage amongst young people so as to tell an interesting story.
\item In general, there has been a decline in ``community'' since about 1970. Coincidentally, this is also when marriage rates start to trend downward.
\item Related to that, I have a couple potential ideas I am looking into:
\begin{itemize}
	\item Video games/leisure -- there is a recent JPE paper about how young men work less because they'd rather play video games. Is this similar for forming relationships?
	\item Increased political polarization. I read something recently about how young men are becoming more conservative while this isn't true for young women. Is this creating friction in matching markets?
\end{itemize}
\item I am currently going through some data to see if I can estimate any reduced-form/first-stage relationships and then I'll try to find some exogenous variation to do something causal.
\end{itemize}

\item \textbf{Endogenous separation}
\begin{itemize}
	\item I have been reading some IO papers to see how they handle this type of problem.
	\item Models of firm entry/firm exit are not too dissimilar from what I am considering.
	\item If exiting is a terminal state, this problem is solved. I can derive a closed-form separation function and identify exit costs if I assume that marriage quality shocks are common to both partners and I have multiple periods of divorce/separation data.
	\item What is novel in my case is the possibility of remarriage. I am thinking about how to deal with this.
\end{itemize}



\end{itemize}


\newpage
\bibliographystyle{plainnat}
%\bibliography{/home/selliott/Research/bib/matching}

\end{document}